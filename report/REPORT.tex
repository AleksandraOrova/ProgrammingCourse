\documentclass[12pt,a4paper]{report}
\usepackage[utf8]{inputenc}
\usepackage[russian]{babel}
\usepackage[OT1]{fontenc}
\usepackage{amsmath}
\usepackage{amsfonts}
\usepackage{amssymb}
\usepackage{graphicx}
\usepackage{cmap}					% поиск в PDF
\usepackage{mathtext} 				% русские буквы в формулах
%\usepackage{tikz-uml}               % uml диаграммы


% TODOs
\usepackage[%
  colorinlistoftodos,
  shadow
]{todonotes}

% Генератор текста
\usepackage{blindtext}

%------------------------------------------------------------------------------

% Подсветка синтаксиса
\usepackage{color}
\usepackage{xcolor}
\usepackage{listings}
 
 % Цвета для кода
\definecolor{string}{HTML}{B40000} % цвет строк в коде
\definecolor{comment}{HTML}{008000} % цвет комментариев в коде
\definecolor{keyword}{HTML}{1A00FF} % цвет ключевых слов в коде
\definecolor{morecomment}{HTML}{8000FF} % цвет include и других элементов в коде
\definecolor{captiontext}{HTML}{FFFFFF} % цвет текста заголовка в коде
\definecolor{captionbk}{HTML}{999999} % цвет фона заголовка в коде
\definecolor{bk}{HTML}{FFFFFF} % цвет фона в коде
\definecolor{frame}{HTML}{999999} % цвет рамки в коде
\definecolor{brackets}{HTML}{B40000} % цвет скобок в коде
 
 % Настройки отображения кода
\lstset{
language=C, % Язык кода по умолчанию
morekeywords={*,...}, % если хотите добавить ключевые слова, то добавляйте
 % Цвета
keywordstyle=\color{keyword}\ttfamily\bfseries,
stringstyle=\color{string}\ttfamily,
commentstyle=\color{comment}\ttfamily\itshape,
morecomment=[l][\color{morecomment}]{\#}, 
 % Настройки отображения     
breaklines=true, % Перенос длинных строк
basicstyle=\ttfamily\footnotesize, % Шрифт для отображения кода
backgroundcolor=\color{bk}, % Цвет фона кода
%frame=lrb,xleftmargin=\fboxsep,xrightmargin=-\fboxsep, % Рамка, подогнанная к заголовку
frame=tblr
rulecolor=\color{frame}, % Цвет рамки
tabsize=3, % Размер табуляции в пробелах
showstringspaces=false,
 % Настройка отображения номеров строк. Если не нужно, то удалите весь блок
numbers=left, % Слева отображаются номера строк
stepnumber=1, % Каждую строку нумеровать
numbersep=5pt, % Отступ от кода 
numberstyle=\small\color{black}, % Стиль написания номеров строк
 % Для отображения русского языка
extendedchars=true,
literate={Ö}{{\"O}}1
  {Ä}{{\"A}}1
  {Ü}{{\"U}}1
  {ß}{{\ss}}1
  {ü}{{\"u}}1
  {ä}{{\"a}}1
  {ö}{{\"o}}1
  {~}{{\textasciitilde}}1
  {а}{{\selectfont\char224}}1
  {б}{{\selectfont\char225}}1
  {в}{{\selectfont\char226}}1
  {г}{{\selectfont\char227}}1
  {д}{{\selectfont\char228}}1
  {е}{{\selectfont\char229}}1
  {ё}{{\"e}}1
  {ж}{{\selectfont\char230}}1
  {з}{{\selectfont\char231}}1
  {и}{{\selectfont\char232}}1
  {й}{{\selectfont\char233}}1
  {к}{{\selectfont\char234}}1
  {л}{{\selectfont\char235}}1
  {м}{{\selectfont\char236}}1
  {н}{{\selectfont\char237}}1
  {о}{{\selectfont\char238}}1
  {п}{{\selectfont\char239}}1
  {р}{{\selectfont\char240}}1
  {с}{{\selectfont\char241}}1
  {т}{{\selectfont\char242}}1
  {у}{{\selectfont\char243}}1
  {ф}{{\selectfont\char244}}1
  {х}{{\selectfont\char245}}1
  {ц}{{\selectfont\char246}}1
  {ч}{{\selectfont\char247}}1
  {ш}{{\selectfont\char248}}1
  {щ}{{\selectfont\char249}}1
  {ъ}{{\selectfont\char250}}1
  {ы}{{\selectfont\char251}}1
  {ь}{{\selectfont\char252}}1
  {э}{{\selectfont\char253}}1
  {ю}{{\selectfont\char254}}1
  {я}{{\selectfont\char255}}1
  {А}{{\selectfont\char192}}1
  {Б}{{\selectfont\char193}}1
  {В}{{\selectfont\char194}}1
  {Г}{{\selectfont\char195}}1
  {Д}{{\selectfont\char196}}1
  {Е}{{\selectfont\char197}}1
  {Ё}{{\"E}}1
  {Ж}{{\selectfont\char198}}1
  {З}{{\selectfont\char199}}1
  {И}{{\selectfont\char200}}1
  {Й}{{\selectfont\char201}}1
  {К}{{\selectfont\char202}}1
  {Л}{{\selectfont\char203}}1
  {М}{{\selectfont\char204}}1
  {Н}{{\selectfont\char205}}1
  {О}{{\selectfont\char206}}1
  {П}{{\selectfont\char207}}1
  {Р}{{\selectfont\char208}}1
  {С}{{\selectfont\char209}}1
  {Т}{{\selectfont\char210}}1
  {У}{{\selectfont\char211}}1
  {Ф}{{\selectfont\char212}}1
  {Х}{{\selectfont\char213}}1
  {Ц}{{\selectfont\char214}}1
  {Ч}{{\selectfont\char215}}1
  {Ш}{{\selectfont\char216}}1
  {Щ}{{\selectfont\char217}}1
  {Ъ}{{\selectfont\char218}}1
  {Ы}{{\selectfont\char219}}1
  {Ь}{{\selectfont\char220}}1
  {Э}{{\selectfont\char221}}1
  {Ю}{{\selectfont\char222}}1
  {Я}{{\selectfont\char223}}1
  {і}{{\selectfont\char105}}1
  {ї}{{\selectfont\char168}}1
  {є}{{\selectfont\char185}}1
  {ґ}{{\selectfont\char160}}1
  {І}{{\selectfont\char73}}1
  {Ї}{{\selectfont\char136}}1
  {Є}{{\selectfont\char153}}1
  {Ґ}{{\selectfont\char128}}1
  {\{}{{{\color{brackets}\{}}}1 % Цвет скобок {
  {\}}{{{\color{brackets}\}}}}1 % Цвет скобок }
}
 
 % Для настройки заголовка кода
\usepackage{caption}
\DeclareCaptionFont{white}{\color{сaptiontext}}
\DeclareCaptionFormat{listing}{\parbox{\linewidth}{\colorbox{сaptionbk}{\parbox{\linewidth}{#1#2#3}}\vskip-4pt}}
\captionsetup[lstlisting]{format=listing,labelfont=white,textfont=white}
\renewcommand{\lstlistingname}{Код} % Переименование Listings в нужное именование структуры


%------------------------------------------------------------------------------

\author{А.~Д.~Орова}
\title{Программирование}
\begin{document}

\listoftodos
\maketitle
\tableofcontents{}

\chapter{Основные конструкции языка}
%############################################################
\section{1. Банковская задача}
\subsection{Задание}
\hspace{\parindent}
Человек положил в банк сумму в s рублей под p\% годовых (проценты начисляются в конце года). Сколько денег будет на счету через 5 лет? 

\subsection{Теоретические сведения}
\hspace{\parindent}
Для решения данной задачи используется формула вычисления сложного процента: \begin{displaymath} S = x + (1 + P)^{n}  \end{displaymath}, где $S$ - конечная сумма, $x$ - начальная сумма,$P$ - процентная ставка и $n$ - количество кварталов (лет).

Для реализации данного алгоритма был использован цикл for, счетчиком которого является
количество лет, данное в задании. Также были применены
\todo[inline]{функции стандартной библиотеки, описанные в заголовочном файле stdio.h и всюду это исправьте}
  функции библиотек stdio для ввода и вывода информации и math для выполнения необходимых вычислений.


\subsection{Проектирование}
\hspace{\parindent}
В ходе проектирования было решено выделить две функции.
\todo[inline]{здесь лучше указывать полный прототип фукнции со списком параметров и типом возвращаемого значения}

	\begin{enumerate}
		\item \verb+bank+	
		 Функция вычисляет конечную сумму денег по вкладу в банк на 5 лет при определенном проценте, передаваемым в программу пользователем.
		 Параметрами функции являются две переменные типа float: $summa$ и $percent$. В первую переменную передается первоначальная сумма, которую пользователь желает положить в банк, во вторую - процент, под который кладутся деньги.	 
		\item \verb+bank_console_ui+	 
		 В этой функции реализованно взаимодействие с пользователем. В ней выполняется считывание 2 значений из консоли. Если данные введены правильно, то выполняется функция b, аргументами которой являются данные введенные пользователем, затем в зависимости от значения, которое вернула эта функция, в  консоль выводится соответствющее сообщение. 		
	\end{enumerate}
	
\subsection{Описание тестового стенда и методики тестирования}
\hspace{\parindent}
\todo[inline]{Вы же сами пишете, что используете Ubuntu, а MinGW -- это inimalist GNU for Windows -- нестыковочки...}
Среда разработки QtCreator 3.5.0, компилятор Qt 5.5.0 MinGW 32bit, операционная система Ubuntu 14.04. Были проведены ручные, а также модульные тесты.

\subsection{Тестовый план и результаты тестирования}
\begin{enumerate}
\item \textbf{Ручные тесты}
\begin{description}
\item[I тест]
\hspace{\parindent}
\begin{flushleft}
\todo[inline]{здесь не понятно, чему соответствует 1000, а чему 20}
\begin{description}
\item[Входные данные:] 1000 20
\item[Выходные данные:] 2488,32
\item[Результат:] Тест успешно пройден
\end{description}
\end{flushleft}
\end{description}

\begin{description}
\item[II тест]
\hspace{\parindent}
\begin{flushleft}
\begin{description}
\item[Входные данные:] 100 15
\item[Выходные данные:] 201,13
\item[Результат:] Тест успешно пройден
\end{description}
\end{flushleft}
\end{description}

\item \textbf{Модульные тесты \textit{Qt}}
\begin{description}
\item[I тест]
\hspace{\parindent}
\begin{flushleft}
\begin{description}
\item[Входные данные:] 200 25
\item[Выходные данные:] 610,35
\item[Результат:] Тест успешно пройден
\end{description}
\end{flushleft}
\end{description}

\begin{description}
\item[II тест]
\hspace{\parindent}
\begin{flushleft}
\begin{description}
\item[Входные данные:] 10 90
\item[Выходные данные:] 247,60
\item[Результат:] Тест успешно пройден
\end{description}
\end{flushleft}
\end{description}

\end{enumerate}

\subsection{Выводы}
\hspace{\parindent}
\todo[inline]{Следует использовать обезличенные конструкции (было отработано, получен опыт), или в третьем лице (автор отработал, автором был получен опыт)}
При выполнении задания я отработала свои навыки в работе с основными конструкциями языка и получила опыт в организации функций одной программы.
\subsection*{Листинги}
\begin{itemize}
\item[] \verb-bank.c-
\lstinputlisting[]
{../sources/SubDirrrProject/lib/bank.c}
\item[] \verb-bank_console_ui.c-
\lstinputlisting[]
{../sources/SubDirrrProject/app/bank_console_ui.c}
\item[] \verb-bank.h.c-
\lstinputlisting[]
{../sources/SubDirrrProject/lib/bank.h}
\item[] \verb-bank_console_ui.h-
\lstinputlisting[]
{../sources/SubDirrrProject/app/bank_console_ui.h}
\end{itemize}

%\todo[inline]{Не забыть вставить все исходники}
%##################################################################################################################################################
%
%##################################################################################################################################################

\section{Задание}
\subsection{2. Возможность расположения домов на участке}
\hspace{\parindent}
Определить, можно ли на прямоугольном участке застройки размером а на b метров разместить 2 дома размером р на q и r на s метров? Дома можно располагать только параллельно сторонам участка.

\subsection{Теоретические сведения}
\hspace{\parindent}
Для решения данной задачи необходимо знать, поместятся ли 2 дома на участке, и на каком участке. То есть если один дом не будет перекрывать другой, также находящийся на данной терртории, то программа выдаст положительный ответ. Иначе, если физически невозможно
 расположить 2 дома на однй территории, то программа выдаст отрицательный ответ. \todo[inline]{автором, или просто была}Мной была использована конструкция if...else. А также  функции библеотек stdio для ввода и вывода.

\subsection{Проектирование}
\hspace{\parindent}
Для решения данной задачи я использую 6 переменных, в каждую из которых передаю линейную характеристику дома.
	В ходе проектирования были выделены следущие функции:
\begin{enumerate}
	\item \textbf{home}
	В  функции выполняется проверка того, могут ли быть два конкретных дома поместиться на конкретном  участке. Проверка необходимо, так как в некоторых случаях два дома могут перекрывать участки друг друга. Условие состоит в том, чтобы такого перекрытия не было.
	Параметрами функции являются шесть переменных типа int. Если аргументы соответсвуют условию, то функциия вернет 1, в противном случае функция вернет 0.
		\item \textbf{home\_console\_ui}
	Функция реализует взаимодействие с пользователем, который вводит длины домов. В случае, если предыдущая функция возвращает 1, то данная функция выведет на экран Yes, иначе No.
\end{enumerate}

\subsection{Описание тестового стенда и методики тестирования}
\hspace{\parindent}
Среда разработки QtCreator 3.5.0, компилятор Qt 5.5.0 MinGW 32bit, операционная система Ubuntu 14.04. Были проведены ручные, а также модульные тесты.

\subsection{Тестовый план и результаты тестирования}
\begin{enumerate}
\item \textbf{Ручные тесты}
\begin{description}
\item[I тест]
\hspace{\parindent}
\begin{flushleft}
\begin{description}
\item[Входные данные:] 50 60 40 30 40 40
\item[Выходные данные:] No
\item[Результат:] Тест успешно пройден
\end{description}
\end{flushleft}
\end{description}

\begin{description}
\item[II тест]
\hspace{\parindent}
\begin{flushleft}
\begin{description}
\item[Входные данные:] 90 90 70 40 30 80
\item[Выходные данные:] Yes
\item[Результат:] Тест успешно пройден
\end{description}
\end{flushleft}
\end{description}

\item \textbf{Модульные тесты \textit{Qt}}
\begin{description}
\item[I тест]
\hspace{\parindent}
\begin{flushleft}
\begin{description}
\item[Входные данные:] 40 70 30 30 30 30
\item[Выходные данные:] Yes
\item[Результат:] Тест успешно пройден
\end{description}
\end{flushleft}
\end{description}

\begin{description}
\item[II тест]
\hspace{\parindent}
\begin{flushleft}
\begin{description}
\item[Входные данные:] 80 30 40 50 20 20 
\item[Выходные данные:] No
\item[Результат:] Тест успешно пройден
\end{description}
\end{flushleft}
\end{description}

\end{enumerate}

\subsection{Выводы}
\hspace{\parindent} 
При выполнении задания я научилась использовать конструкцию if...else для решения не совсем тривиальных задач.
\subsection*{Листинги}
\begin{itemize}
\item[] \verb-home.c-
\lstinputlisting[]
{../sources/SubDirrrProject/lib/home.c}
\item[] \verb-home_console_ui.c-
\lstinputlisting[]
{../sources/SubDirrrProject/app/home_console_ui.c}
\item[] \verb-home.h-
\lstinputlisting[]
{../sources/SubDirrrProject/lib/home.h}
\item[] \verb-home_console_ui.h-
\lstinputlisting[]
{../sources/SubDirrrProject/app/home_console_ui.h}
\end{itemize}
%##################################################################################################################################################
%
%##################################################################################################################################################
\chapter{Циклы}
\section{Таблица перевода из дюймов в сантиметры}
\subsection{Задание}
\hspace{\parindent}
Вывести на экран таблицу пересчета сантиметров в дюймы и обратно до заданного расстояния в сантиметрах, по возрастанию расстояний, как указано в примере (1 дюйм = 2.54 см). Пример для 6 см:
	
\begin{center}
\begin{tabular}{ccc}
	дюймы  &  см \\
	0.39 &  1.00 \\
	0.79 &  2.00 \\
	1.00 &  2.54 \\
	1.18 &  3.00 \\
	1.57 &  4.00 \\
	1.97 &  5.00 \\
	2.00 &  5.08 \\
	2.36 &  6.00\\
\end{tabular}
\end{center}	
	
\subsection{Теоретические сведения}
\hspace{\parindent}
Для того, чтобы перевести из сантиметров в дюймы необходимо количество сантиметров поделить на эквивалент, равный 2,54. Для того чтобы перевести из дюймов в сантиметры - соответственно умножить на 2,54.
	Для выполнения задания использоваилсь функции библиотеки stdio для ввода и вывода.
	
\subsection{Проектирование}
\hspace{\parindent}
В ходе проектирования было выделено три функции:
\begin{enumerate}
	\item \textbf{cm\_to\_inch}
	Функция возвращает переданное ей количество сантиметров, поделенное на эквивалент.
	Параметром функции является переменная типа float.
	\item \textbf{inch\_to\_cm}
	Функция возвращает переданное ей количество дюймов, поделенное на эквивалент.
	Параметром функции является переменная типа float.	
	\item \textbf{inch\_to\_cm\_console\_ui}
	В  Функции реализованно взаимодействие с пользователем. В ней выполняется считывание из консоли числа, равного количеству сантиметров, котоое пользователь хочет перевести в сантиметры. Пользователю на экран выводится таблица от 1 сантиметра до введенного значения.	
\end{enumerate}

\subsection{Описание тестового стенда и методики тестирования}
\hspace{\parindent}
Среда разработки QtCreator 3.5.0, компилятор Qt 5.5.0 MinGW 32bit, операционная система Ubuntu 14.04. Были проведены ручные, а также модульные тесты.

\subsection{Тестовый план и результаты тестирования}
\hspace{\parindent}
\begin{enumerate}
\item \textbf{Ручные тесты}
\begin{description}
\item[I тест]
\hspace{\parindent}
\begin{flushleft}
\begin{description}
\item[Входные данные:] 3
\item[Выходные данные:]  \verb/"0,39\t 1,00\n 0,79\t 2,00\n 1,00\t 2,54\n 1,18\t 3,00\n"/ 
\item[Результат:] Тест успешно пройден
\end{description}
\end{flushleft}
\end{description}

\begin{description}
\item[II тест]
\hspace{\parindent}
\begin{flushleft}
\begin{description}
\item[Входные данные:] 4
\item[Выходные данные:] \verb/"0,39\t 1,00\n 0,79\t 2,00\n 1,00\t 2,54\n 1,18\t 3,00\n 1,57\t 4,00\n"/ 
\item[Результат:] Тест успешно пройден
\end{description}
\end{flushleft}
\end{description}

\item \textbf{Модульные тесты \textit{Qt}}
\begin{description}
\item[I тест]
\hspace{\parindent}
\begin{flushleft}
\begin{description}
\item[Входные данные:] 2
\item[Выходные данные:] \verb/"0,39\t 1,00\n 0,79\t 2,00\n"/ 
\item[Результат:] Тест успешно пройден
\end{description}
\end{flushleft}
\end{description}

\begin{description}
\item[II тест]
\hspace{\parindent}
\begin{flushleft}
\begin{description}
\item[Входные данные:] 5
\item[Выходные данные:] \verb/"0,39\t 1,00\n 0,79\t/
\verb/2,00\n 1,00\t 2,54\n 1,18\t 3,00\n 1,57\t 4,00\n 1.97\t 5,00"/ 
\item[Результат:] Тест успешно пройден
\end{description}
\end{flushleft}
\end{description}

\end{enumerate}
\subsection{Выводы}
\hspace{\parindent}
В ходе выполнения я отработала навыки работы с циклом с предусловием. 
\subsection*{Листинги}
\begin{itemize}
\item[] \verb-cm_to_inch.c-
\lstinputlisting[]
{../sources/SubDirrrProject/lib/cm_to_inch.c}
\item[] \verb-cm_to_inch_console_ui.c-
\lstinputlisting[]
{../sources/SubDirrrProject/app/cm_to_inch_console_ui.c}
\item[] \verb-cm_to_inch.h-
\lstinputlisting[]
{../sources/SubDirrrProject/lib/cm_to_inch.h}
\item[] \verb-cm_to_inch_console_ui.h-
\lstinputlisting[]
{../sources/SubDirrrProject/app/cm_to_inch_console_ui.h}
\end{itemize}

%##################################################################################################################################################
%
%##################################################################################################################################################
\chapter{Массивы}
\section{Заполнение матрицы по спирали}
\subsection{Задание}
\hspace{\parindent}
Матрицу $A(m,n)$ заполнить натуральными числами от 1 до $m \times n$ по спирали, начинающейся в левом верхнем углу и закрученной по часовой стрелке.

\subsection{Теоретические сведения}
\hspace{\parindent}
Для выполнения задания использовался цикл for, конструкция if...else, а также функции бибилиотек stdlib для динамического выделения и освобождения памяти, stdio для ввода, вывода информации и работы с файлами.

\subsection{Проектирование}
\hspace{\parindent} 
В ходе проектирования были выделены четыре функции:
\begin{enumerate}
	\item \textbf{initializeMatrix}
	Функция считывает из файла заданное количество целых чисел и сохраняет их в массив.
	Параметрами функции являются символьная строку содержащая имя файла, массив типа int, куда будут сохранятся считанные числа и переменная типа int, содержащая количество элемнтов.
	Функция возвращает количество успешно считанных из файла значений. 
	\item \textbf{fillSpiralMatrix}	
	С помощью цикла с предусловием for() инкремента "++", а также конструкции if...else функция заполняет двумерный массив как и необходимо в задании, то есть по спирали. 
	\item \textbf{matrix\_console\_ui}
	Функция открывает файл и закрывает его после всех действий, считывает размеры матрицы(двумерного массива), выделяет память и позже её освобождает. Основная цель данной фунцкии - взаимодействие с пользователем.
	\item \textbf{printMatrix}	
	Функция выводит матрицу, заполненную по спирали на экран в консоль.
\end{enumerate}
\subsection{Описание тестового стенда и методики тестирования}
Среда разработки QtCreator 3.5.0, компилятор Qt 5.5.0 MinGW 32bit, операционная система Ubuntu 14.04. Были проведены ручные, а также модульные тесты.

\subsection{Тестовый план и результаты тестирования}
\hspace{\parindent}
\begin{enumerate}
\item \textbf{Модульные тесты \textit{Qt}}

\begin{description}
\item[I тест]
\hspace{\parindent}
\begin{flushleft}
\begin{description}
\item[Входные данные:] 5 7
\item[Выходные данные:]
\hspace{\parindent}
\begin{flushleft}
 1  2  3  4  5 

20 21 22 23  6

19 32 33 24  7

18 31 34 25  8

17 30 35 26  9

16 29 28 27 10

15 14 13 12 11
\end{flushleft}
\item[Результат:] Тест успешно пройден
\end{description}
\end{flushleft}
\end{description}
\todo[inline]{Только с этого задания начали использовать cppcheck? и версию не указываете...}
\item \textbf{Статический анализ с использование утилиты \textit{cppcheck}}

Утилита \textit{cppcheck} не выдала никаких предупреждений.
\end{enumerate}
\subsection{Выводы}
\hspace{\parindent}
При выполнении задания я поняла принцип организации программы при работе с выделением динамической памяти, научилася работать с файлами.
\subsection*{Листинги}
\begin{itemize}
\item[] \verb-matrix.c-
\lstinputlisting[]
{../sources/SubDirrrProject/lib/matrix.c}
\item[] \verb-matrix_console_ui.c-
\lstinputlisting[]
{../sources/SubDirrrProject/app/matrix_console_ui.c}
\item[] \verb-matrix.h-
\lstinputlisting[]
{../sources/SubDirrrProject/lib/matrix.h}
\item[] \verb-matrix_console_ui.h-
\lstinputlisting[]
{../sources/SubDirrrProject/app/matrix_console_ui.h}
\end{itemize}


%##################################################################################################################################################
%
%##################################################################################################################################################
\chapter{Строки}
\section{Выравнивание по ширине}
\subsection{Задание}
\hspace{\parindent}
Для реализации данной задачи было решено создать некоторое количество 
\subsection{Проектирование}
%Какие функции было решено выделить, какие у этих функций контракты, как организовано взаимодействие с пользователем (чтение/запись из консоли, из файла, из параметров командной строки), форматы файлов и др.
\hspace{\parindent}
В ходе проектирования было решено выделить 5 функций, 2 из которых отвечают за логику, а остальные -- за взаимодействие с пользователем.
\begin{enumerate}

	\item \textbf{cm\_to\_inch\_console\_ui}
	Функция для взаимодействия пользователем.
	
	\item \textbf{initialize\_text}
	Функция инициализирует введенный текст, а также выделяет на него память.

	\item \textbf{initialize\_string}
	Функция инициализирует переданную ей строку.
	
	\item \textbf{input\_text}
	Функция считывает строку.
	
	\item \textbf{print\_text}
	Функция выводит на экран текст.
	
	\item \textbf{get\_length}
	Функция считывает длину строки.
	
	\item \textbf{count\_spaces}
	Функция считает количество пробелов в строке.
	
	\item \textbf{count\_chars}
	Функция считает количество символов в строке.
	
	\item \textbf{get\_max\_string\_length}
	Функция ищет среди строк самую длинную.
	
	\item \textbf{insert\_char}
	Функция вставляет символ.
	
	\item \textbf{insert\_chars}
	Функция вставляет символы.
	
	\item \textbf{get\_string}
	Функция считывающая строку (массив).
	
	\item \textbf{get\_char\_index}
	Функция, возвращающая индекс num-ового вхождения символа chr в строку str.
	
	\item \textbf{spread\_text}
	Функция, которая работает с пробелами.	

\end{enumerate}
\subsection{Описание тестового стенда и методики тестирования}
Среда разработки QtCreator 3.5.0, компилятор Qt 5.5.0 MinGW 32bit, операционная система Ubuntu 14.04. Были проведены ручные, а также модульные тесты.

\subsection{Тестовый план и результаты тестирования}
\hspace{\parindent}
\begin{enumerate}
\item \textbf{Модульные тесты \textit{Qt}}

\begin{description}
\item[I тест]
\hspace{\parindent}
\begin{flushleft}
\begin{description}
\item[Входные данные:] \verb-"    dgrtf tfhrna f htya"- 


\item[Выходные данные:] \verb-"dgrtf   tfhrna  f   htya"-

\item[Результат:] Тест успешно пройден
\end{description}
\end{flushleft}
\end{description}

\item \textbf{Статический анализ \textit{cppcheck}}

Утилита \textit{cppcheck} не выдала никаких предупреждений.
\end{enumerate}

\subsection{Выводы}
\hspace{\parindent}
В ходе работы я получила опыт в обработке строк, а также укрепила навык работы с файлами.
\subsection*{Листинги}

\begin{itemize}
\item[] \verb-strings.c-
\lstinputlisting[]
{../sources/SubDirrrProject/lib/strings.c}
\item[] \verb-strings_console_ui.c-
\lstinputlisting[]
{../sources/SubDirrrProject/app/strings_console_ui.c}
\item[] \verb-strings.h-
\lstinputlisting[]
{../sources/SubDirrrProject/lib/matrix.h}
\item[] \verb-strings_console_ui.h-
\lstinputlisting[]
{../sources/SubDirrrProject/app/matrix_console_ui.h}
\end{itemize}
%################################################################################################################
%
%################################################################################################################

\chapter{Приложение к главам 1 - 4}

\section{Листинги}
\begin{itemize}
\item[] \verb-main.c-
\lstinputlisting[]
{../sources/SubDirrrProject/app/main.c}
\item[] \verb-main.c-
\lstinputlisting[]
{../sources/SubDirrrProject/test/tst_testtest.cpp}
\end{itemize}
%################################################################################################################
%
%################################################################################################################

\chapter{Введение в классы С++}
\section{Задание 1. Инкапсуляция. Множество}
\subsection{Задание}
\hspace{\parindent}
Реализовать класс МНОЖЕСТВО (целых чисел). Требуемые методы: конструктор, деструктор, копирование, сложение множеств, пересечение множеств, добавление в множество, включение в множество.
\subsection{Теоретические сведения}
\hspace{\parindent}
При разработке приложения была задействована объектная ориентированность языка C++. 

\subsection{Проектирование}
\hspace{\parindent}
В ходе проектирования было решено выделить 2 класса: set и Node.

\hspace{\parindent}
Были выделены методы: \textit{set()} - конструктор, \~set()- деструктор, \textit{copy(set source\&)} - конструтор копирования, \textit{add(set added)} - сложение множеств, \textit{contains(set s)} - пересечение множеств, \textit{intersect(set s)} - добавление в множество, \textit{intersect(set s)} - включение в множество.
Также были выделены вспомогательные методы.


\subsection{Описание тестового стенда и методики тестирования}
Среда разработки QtCreator 3.5.0, компилятор Qt 5.5.0 MinGW 32bit, операционная система Ubuntu 14.04. Были проведены ручные, а также модульные тесты.
\subsection{Тестовый план и результаты тестирования}
ДОДЕЛАТЬ
\subsection{Выводы}
\hspace{\parindent}
Я познакомилася с языком С++. Познакомилася с новой парадигмой программирования - ООП.
\todo[inline]{Ведь не со всех парадигмой, а только с инкапсуляцией}
\subsection*{Листинги}
\begin{itemize}
\item[] \verb-main.cpp-
\lstinputlisting[]
{../sources/SubDirrrProject/set/main.cpp}
\item[] \verb-node.cpp-
\lstinputlisting[]
{../sources/SubDirrrProject/set/node.cpp}
\item[] \verb-set.cpp-
\lstinputlisting[]
{../sources/SubDirrrProject/set/set.cpp}
\item[] \verb-node.h-
\lstinputlisting[]
{../sources/SubDirrrProject/set/node.h}
\item[] \verb-set.h-
\lstinputlisting[]
{../sources/SubDirrrProject/set/set.h}
\end{itemize}

%################################################################################################################
%
%################################################################################################################

\chapter{Классы С++}
\section{Задание 1. Реализовать классы для всех приложений}
\subsection{Задание}
\hspace{\parindent}
Реализовать классы для всех приложений. Поработать с потоками.
\subsection{Выводы}
\hspace{\parindent}
Получила опыт создания классов. Получила опыт в работе с потоками.
\subsection*{Листинги}
\begin{itemize}

\item[] \verb-bankconsoleuicpp.cpp-
\lstinputlisting[]
{../sources/SubDirrrProject/appCPP/bankconsoleuicpp.cpp}

\item[] \verb-bankconsoleuicpp.h-
\lstinputlisting[]
{../sources/SubDirrrProject/appCPP/bankconsoleuicpp.h}

\item[] \verb-cmtoinchconsoleuicpp.cpp-
\lstinputlisting[]
{../sources/SubDirrrProject/appCPP/cmtoinchconsoleuicpp.cpp}

\item[] \verb-cmtoinchconsoleuicpp.h-
\lstinputlisting[]
{../sources/SubDirrrProject/appCPP/cmtoinchconsoleuicpp.h}

\item[] \verb-homeconsoleuicpp.cpp-
\lstinputlisting[]
{../sources/SubDirrrProject/appCPP/homeconsoleuicpp.cpp}

\item[] \verb-homeconsoleuicpp.h-
\lstinputlisting[]
{../sources/SubDirrrProject/appCPP/homeconsoleuicpp.h}

\item[] \verb-matrixconsoleuicpp.cpp-
\lstinputlisting[]
{../sources/SubDirrrProject/appCPP/matrixconsoleuicpp.cpp}

\item[] \verb-matrixconsoleuicpp.h-
\lstinputlisting[]
{../sources/SubDirrrProject/appCPP/matrixconsoleuicpp.h}

\item[] \verb-stringsconsoleuicpp.cpp-
\lstinputlisting[]
{../sources/SubDirrrProject/appCPP/stringsconsoleuicpp.cpp}

\item[] \verb-stringsconsoleuicpp.h-
\lstinputlisting[]
{../sources/SubDirrrProject/appCPP/stringsconsoleuicpp.h}

\item[] \verb-bankcpp.cpp-
\lstinputlisting[]
{../sources/SubDirrrProject/libCPP/bankcpp.cpp}

\item[] \verb-bankcpp.h-
\lstinputlisting[]
{../sources/SubDirrrProject/libCPP/bankcpp.h}

\item[] \verb-cmtoinchcpp.cpp-
\lstinputlisting[]
{../sources/SubDirrrProject/libCPP/cmtoinchcpp.cpp}

\item[] \verb-cmtoinchcpp.h-
\lstinputlisting[]
{../sources/SubDirrrProject/libCPP/cmtoinchcpp.h}

\item[] \verb-homecpp.cpp-
\lstinputlisting[]
{../sources/SubDirrrProject/libCPP/homecpp.cpp}

\item[] \verb-homecpp.h-
\lstinputlisting[]
{../sources/SubDirrrProject/libCPP/homecpp.h}

\item[] \verb-matrixcpp.cpp-
\lstinputlisting[]
{../sources/SubDirrrProject/libCPP/matrixcpp.cpp}

\item[] \verb-matrixcpp.h-
\lstinputlisting[]
{../sources/SubDirrrProject/libCPP/matrixcpp.h}

\item[] \verb-stringscpp.cpp-
\lstinputlisting[]
{../sources/SubDirrrProject/libCPP/stringscpp.cpp}

\item[] \verb-stringscpp.h-
\lstinputlisting[]
{../sources/SubDirrrProject/libCPP/stringscpp.h}

\item[] \verb-tst_testcpptest.cpp-
\lstinputlisting[]
{../sources/SubDirrrProject/testCPP/tst_testcpptest.cpp}

\end{itemize}
\end{document}