\documentclass[12pt,a4paper]{report}
\usepackage[utf8]{inputenc}
\usepackage[russian]{babel}
\usepackage[OT1]{fontenc}
\usepackage{amsmath}
\usepackage{amsfonts}
\usepackage{amssymb}
\usepackage{graphicx}
\usepackage{cmap}					% поиск в PDF
\usepackage{mathtext} 				% русские буквы в формулах
%\usepackage{tikz-uml}               % uml диаграммы

% TODOs
\usepackage[%
  colorinlistoftodos,
  shadow
]{todonotes}

% Генератор текста
\usepackage{blindtext}

%------------------------------------------------------------------------------

% Подсветка синтаксиса
\usepackage{color}
\usepackage{xcolor}
\usepackage{listings}
 
 % Цвета для кода
\definecolor{string}{HTML}{B40000} % цвет строк в коде
\definecolor{comment}{HTML}{008000} % цвет комментариев в коде
\definecolor{keyword}{HTML}{1A00FF} % цвет ключевых слов в коде
\definecolor{morecomment}{HTML}{8000FF} % цвет include и других элементов в коде
\definecolor{captiontext}{HTML}{FFFFFF} % цвет текста заголовка в коде
\definecolor{captionbk}{HTML}{999999} % цвет фона заголовка в коде
\definecolor{bk}{HTML}{FFFFFF} % цвет фона в коде
\definecolor{frame}{HTML}{999999} % цвет рамки в коде
\definecolor{brackets}{HTML}{B40000} % цвет скобок в коде
 
 % Настройки отображения кода
\lstset{
language=C, % Язык кода по умолчанию
morekeywords={*,...}, % если хотите добавить ключевые слова, то добавляйте
 % Цвета
keywordstyle=\color{keyword}\ttfamily\bfseries,
stringstyle=\color{string}\ttfamily,
commentstyle=\color{comment}\ttfamily\itshape,
morecomment=[l][\color{morecomment}]{\#}, 
 % Настройки отображения     
breaklines=true, % Перенос длинных строк
basicstyle=\ttfamily\footnotesize, % Шрифт для отображения кода
backgroundcolor=\color{bk}, % Цвет фона кода
%frame=lrb,xleftmargin=\fboxsep,xrightmargin=-\fboxsep, % Рамка, подогнанная к заголовку
frame=tblr
rulecolor=\color{frame}, % Цвет рамки
tabsize=3, % Размер табуляции в пробелах
showstringspaces=false,
 % Настройка отображения номеров строк. Если не нужно, то удалите весь блок
numbers=left, % Слева отображаются номера строк
stepnumber=1, % Каждую строку нумеровать
numbersep=5pt, % Отступ от кода 
numberstyle=\small\color{black}, % Стиль написания номеров строк
 % Для отображения русского языка
extendedchars=true,
literate={Ö}{{\"O}}1
  {Ä}{{\"A}}1
  {Ü}{{\"U}}1
  {ß}{{\ss}}1
  {ü}{{\"u}}1
  {ä}{{\"a}}1
  {ö}{{\"o}}1
  {~}{{\textasciitilde}}1
  {а}{{\selectfont\char224}}1
  {б}{{\selectfont\char225}}1
  {в}{{\selectfont\char226}}1
  {г}{{\selectfont\char227}}1
  {д}{{\selectfont\char228}}1
  {е}{{\selectfont\char229}}1
  {ё}{{\"e}}1
  {ж}{{\selectfont\char230}}1
  {з}{{\selectfont\char231}}1
  {и}{{\selectfont\char232}}1
  {й}{{\selectfont\char233}}1
  {к}{{\selectfont\char234}}1
  {л}{{\selectfont\char235}}1
  {м}{{\selectfont\char236}}1
  {н}{{\selectfont\char237}}1
  {о}{{\selectfont\char238}}1
  {п}{{\selectfont\char239}}1
  {р}{{\selectfont\char240}}1
  {с}{{\selectfont\char241}}1
  {т}{{\selectfont\char242}}1
  {у}{{\selectfont\char243}}1
  {ф}{{\selectfont\char244}}1
  {х}{{\selectfont\char245}}1
  {ц}{{\selectfont\char246}}1
  {ч}{{\selectfont\char247}}1
  {ш}{{\selectfont\char248}}1
  {щ}{{\selectfont\char249}}1
  {ъ}{{\selectfont\char250}}1
  {ы}{{\selectfont\char251}}1
  {ь}{{\selectfont\char252}}1
  {э}{{\selectfont\char253}}1
  {ю}{{\selectfont\char254}}1
  {я}{{\selectfont\char255}}1
  {А}{{\selectfont\char192}}1
  {Б}{{\selectfont\char193}}1
  {В}{{\selectfont\char194}}1
  {Г}{{\selectfont\char195}}1
  {Д}{{\selectfont\char196}}1
  {Е}{{\selectfont\char197}}1
  {Ё}{{\"E}}1
  {Ж}{{\selectfont\char198}}1
  {З}{{\selectfont\char199}}1
  {И}{{\selectfont\char200}}1
  {Й}{{\selectfont\char201}}1
  {К}{{\selectfont\char202}}1
  {Л}{{\selectfont\char203}}1
  {М}{{\selectfont\char204}}1
  {Н}{{\selectfont\char205}}1
  {О}{{\selectfont\char206}}1
  {П}{{\selectfont\char207}}1
  {Р}{{\selectfont\char208}}1
  {С}{{\selectfont\char209}}1
  {Т}{{\selectfont\char210}}1
  {У}{{\selectfont\char211}}1
  {Ф}{{\selectfont\char212}}1
  {Х}{{\selectfont\char213}}1
  {Ц}{{\selectfont\char214}}1
  {Ч}{{\selectfont\char215}}1
  {Ш}{{\selectfont\char216}}1
  {Щ}{{\selectfont\char217}}1
  {Ъ}{{\selectfont\char218}}1
  {Ы}{{\selectfont\char219}}1
  {Ь}{{\selectfont\char220}}1
  {Э}{{\selectfont\char221}}1
  {Ю}{{\selectfont\char222}}1
  {Я}{{\selectfont\char223}}1
  {і}{{\selectfont\char105}}1
  {ї}{{\selectfont\char168}}1
  {є}{{\selectfont\char185}}1
  {ґ}{{\selectfont\char160}}1
  {І}{{\selectfont\char73}}1
  {Ї}{{\selectfont\char136}}1
  {Є}{{\selectfont\char153}}1
  {Ґ}{{\selectfont\char128}}1
  {\{}{{{\color{brackets}\{}}}1 % Цвет скобок {
  {\}}{{{\color{brackets}\}}}}1 % Цвет скобок }
}
 
 % Для настройки заголовка кода
\usepackage{caption}
\DeclareCaptionFont{white}{\color{сaptiontext}}
\DeclareCaptionFormat{listing}{\parbox{\linewidth}{\colorbox{сaptionbk}{\parbox{\linewidth}{#1#2#3}}\vskip-4pt}}
\captionsetup[lstlisting]{format=listing,labelfont=white,textfont=white}
\renewcommand{\lstlistingname}{Код} % Переименование Listings в нужное именование структуры


%------------------------------------------------------------------------------

\author{А.~Д.~Орова}
\title{Программирование}
\begin{document}
\maketitle
\chapter{Основные конструкции языка}
%############################################################
\section{Задание 1}
\subsection{Задание}
Человек положил в банк сумму в s рублей под p\% годовых (проценты начисляются в конце года). Сколько денег будет на счету через 5 лет? 
\subsection{Теоретические сведения}

Для решения данной задачи используется формула вычисления сложного процента: \begin{displaymath} S = x + (1 + P)^{n}  \end{displaymath}, где $S$ - конечная сумма, $x$ - начальная сумма,$P$ - процентная ставка и $n$ - количество кварталов (лет).

Для реализации данного алгоритма был использован цикл for, счетчиком которого является количество лет, данное в задании. Также были применены функции библиотек stdio для ввода и вывода информации и math для выполнения необходимых вычислений.


\subsection{Проектирование}
В ходе проектирования было решено выделить две функции.
	\begin{enumerate}
		\item bank
		 Функция вычисляет конечную сумму денег по вкладу в банк на 5 лет при определенном проценте, передаваемым в программу пользователем.
		 Параметрами функции являются две переменные типа float: $summa$ и $percent$. В первую переменную передается первоначальная сумма, которую пользователь желает положить в банк, во вторую - процент, под который кладутся деньги.	 
		\item bankСonsoleUI 
		 В этой функции реализованно взаимодействие с пользователем. В ней выполняется считывание 2 значений из консоли. Если данные введены правильно, то выполняется функция b, аргументами которой являются данные введенные пользователем, затем в зависимости от значения, которое вернула эта функция, в  консоль выводится соответствющее сообщение. 		
	\end{enumerate}


\subsection{Описание тестового стенда и методики тестирования}
Среда разработки QtCreator 3.5.0, компилятор Qt 5.5.0 MinGW 32bit, операционная система Ubuntu 14.04.
Для тестирования работы программы были выполнены статический и динамический анализ, также было проведено автоматической тестирование.

\subsection{Тестовый план и результаты тестирования}
		При автоматическом тестировании функция вызывалась несколько раз с разными наборами аргументов, затем значения, возвращенные функцией, и полученные орни сравнивались с ожидаемыми значениями. Результаты тестирования представлены в таблице:  
	 

 \begin{tabular}{|c|c|c|c|c|}
	\hline 	
	Входные данные& Ожидаемый результат & Полученнный результат \\
	\hline
	1000 20 & 2488.32 & 2488.32 \\
	\hline
 \end{tabular}
\subsection{Выводы}
При выполнении задания я отработала свои навыки в работе с основными конструкциями языка и получила опыт в организации функций одной программы.

%\subsection*{Листинги}

%bank
%\lstinputlisting[]
%{/home/aleksandra/ProgrammingCourse/sources/SubDirrrProject/librces/SubDirrrProject/lib/bank.c}

%bank_console_ui
%\lstinputlisting[]
%{../sources/Sub_Dirrr_Project/app/bank_console_ui.c}

%############################################################

\section{Задание 2}
\subsection{Задание}
Определить, можно ли на прямоугольном участке застройки размером а на b метров разместить 2 дома размером р на q и r на s метров? Дома можно располагать только параллельно сторонам участка.
	
\subsection{Теоритические сведения}
Для решения данной задачи необходимо знать, поместятся ли 2 дома на участке, и на каком участке. То есть если один дом не будет перекрывать другой, также находящийся на данной терртории, то программа выдаст положительный ответ. Иначе, если физически невозможно расположить 2 дома на однй территории, то программа выдаст отрицательный ответ. Мной была использована конструкция if...else. А также  функции библеотек stdio для ввода и вывода.
	
\subsection{Проектирование}
	Для решения данной задачи я использую 6 переменных, в каждую из которых передаю линейную характеристику дома.
	В ходе проектирования были выделены следущие функции:
\begin{enumerate}
	\item home
	В  функции выполняется проверка того, могут ли быть два конкретных дома поместиться на конкретном  участке. Проверка необходимо, так как в некоторых случаях два дома могут перекрывать участки друг друга. Условие состоит в том, чтобы такого перекрытия не было.
	Параметрами функции являются шесть переменных типа int. Если аргументы соответсвуют условию, то функциия вернет 1, в противном случае функция вернет 0.
		\item homeСonsoleUI
	Функция реализует взаимодействие с пользователем, который вводит длины домов. В случае, если предыдущая функция возвращает 1, то данная функция выведет на экран Yes, иначе No.
\end{enumerate}

\subsection{Описание тестового стенда и методики тестирования}
Среда разработки QtCreator 3.5.0, компилятор Qt 5.5.0 MinGW 32bit, операционная система Ubuntu 14.04.

Для тестирования работы программы были выполнены статический и динамический анализ, также было проведено автоматической тестирование.

\subsection{Тестовый план и результаты тестирования}
В ходе автоматического тестирования функции sumDate подавались пары структур date, затем полученная в ходе выполнения функции структура сравнивалась со структорой, в который были записаны ожидаемые значения.

 \begin{tabular}{|c|c|c|c|c|}
	\hline 	
	Входные данные & Ожидаемый результат & Полученнный результат \\
	\hline
	100 100 23 32 12 35 & No & No \\
	\hline
 \end{tabular}

\subsection{Выводы}
При выполнении задания я научилась использовать конструкцию if...else для решения не совсем тривиальных задач.

%\subsection*{Листинги}
%home
%\lstinputlisting[]
%{../sources/Sub_Dirrr_Project/lib/home.c}


%home_console_ui
%\lstinputlisting[]
%{../sources/Sub_Dirrr_Project/app/home_console_ui.c}



%############################################################
\chapter{Циклы}
\section{Задание 1}
\subsection{Задание}
Вывести на экран таблицу пересчета сантиметров в дюймы и обратно до заданного расстояния в сантиметрах, по возрастанию расстояний, как указано в примере (1 дюйм = 2.54 см). Пример для 6 см:
	
 \begin{tabular}{|c|c|c|c|c|}
	\hline 
 	дюймы  &  см \\
	\hline   
 	0.39 &  1.00 \\
 	\hline   
 	0.79 &  2.00 \\
 	\hline   
 	1.00 &  2.54 \\
 	\hline   
 	1.18 &  3.00 \\
 	\hline   
 	1.57 &  4.00 \\
 	\hline   
 	1.97 &  5.00 \\
 	\hline   
 	2.00 &  5.08 \\
 	\hline   
 	2.36 &  6.00\\
	\hline
 \end{tabular}	
	
\subsection{Теоритические сведения}
Для того, чтобы перевести из сантиметров в дюймы необходимо количество сантиметров поделить на эквивалент, равный 2,54. Для того чтобы перевести из дюймов в сантиметры - соответственно умножить на 2,54.
	Для выполнения задания использоваилсь функции библиотеки stdio для ввода и вывода.
\subsection{Проектирование}
В ходе проектирования было выделено три функции:
\begin{enumerate}
	\item cmToInch
	Функция возвращает переданное ей количество сантиметров, поделенное на эквивалент.
	Параметром функции является переменная типа float.
	\item inchToCm
	Функция возвращает переданное ей количество дюймов, поделенное на эквивалент.
	Параметром функции является переменная типа float.	
	\item inchToCmConsoleUI
	В  Функции реализованно взаимодействие с пользователем. В ней выполняется считывание из консоли числа, равного количеству сантиметров, котоое пользователь хочет перевести в сантиметры. Пользователю на экран выводится таблица от 1 сантиметра до введенного значения.	
\end{enumerate}
\subsection{Описание тестового стенда и методики тестирования}
Среда разработки QtCreator 3.5.0, компилятор Qt 5.5.0 MinGW 32bit, операционная система Ubuntu 14.04.

Для тестирования работы программы были выполнены статический и динамический анализ, также было проведено автоматической тестирование.
\subsection{Тестовый план и результаты тестирования}
Для тестирования программы в подпроект test были добавлены модульные тесты. 
 
 
 
 \begin{tabular}{|c|c|c|}
 	\hline 	
 	Входные данные & Результат & Ожидаемый  результат \\
 	\hline
	3 & 0.39 & 0.39 \\
 	\hline
 	
 \end{tabular}

\subsection{Выводы}
В ходе выполнения я отработала навыки работы с циклом с предусловием. 

%\subsection*{Листинги}
%cm_to_inch
%inch_to_cm
%\lstinputlisting[]
%{../sources/SubDirrrProject/lib/cm_to_inch.c}

%cm_to_inch_console
%\lstinputlisting[]
%{../sources/SubDirrrProject/app/m_to_inch_console_ui.c}

\chapter{Массивы}
\section{Задание 1}
\subsection{Задание}
Матрицу A(m,n) заполнить натуральными числами от 1 до m*n по спирали, начинающейся в левом верхнем углу и закрученной по часовой стрелке.
\subsection{Теоритические сведения}
	Для выполнения задания использовался цикл for, конструкция if...else, а также функции бибилиотек stdlib для динамического выделения и освобождения памяти, stdio для ввода, вывода информации и работы с файлами.
\subsection{Проектирование}
	Задача реализована с помощью нескольких функций:
\begin{enumerate}
	\item matrixConsoleUI
	Ввод и вывод данных реализован с помощью файлов. Входной файл должен содержать два целых числа, записанных через пробел. Выходные файлы создаются по ходу программы и так же содержат целые числа, записанные через пробел. 
	Функция работает с памятью, используюя функции библиотеки stdlib.
	\item printMatrix
	Функция вывода матрицы на экран.
	\item initializeMatrix
	Функция считывает из файла заданное количество целых чисел и сохраняет их в массив.
	Параметрами функции являются символьная строку содержащая имя файла, массив типа int, куда будут сохранятся считанные числа и переменная типа int, содержащая количество элемнтов.
	\item fillSpiralMatrix	
	Функция заполняет матрицу по спирали.
\end{enumerate}
\subsection{Описание тестового стенда и методики тестирования}
Среда разработки QtCreator 3.5.0, компилятор Qt 5.5.0 MinGW 32bit, операционная система Ubuntu 14.04

Для тестирования работы программы были выполнены статический и динамический анализ, также было проведено автоматической тестирование.
\subsection{Тестовый план и результаты тестирования}
Для тестирования программы в подпроект test были добавлены модульные тесты. 


 \begin{tabular}{|c|c|c|}
 	\hline 	
 	Входные данные & Результат & Ожидаемый  результат \\
 	\hline
	3 5 &  1  2 3 &  1  2 3 \\
	& 12 13 4 & 12 13 4 \\
	& 11 14 5 & 11 14 5 \\
	& 10 15 6 & 10 15 6 \\
	&  9  8 7 &  9  8 7 \\   
 	\hline
 \end{tabular}

\subsection{Выводы}
При выполнении задания я поняла принцип организации программы при работе с выделением динамической памяти, научилася работать с файлами.

%\subsection*{Листинги}
%matrix
%\lstinputlisting[]
%{../sources/SubDirrrProject/lib/matrix.c}

%matrix_console_UI
%\lstinputlisting[]
%{../sources/SubDirrrProject/app/matrix_console_UI.c}

\chapter{Строки}
\section{Задание 1}
\subsection{Задание}
Текст, состоящий из ряда строк, выровнять по правому краю так, чтобы каждая строка завершалась непробельным символом (буква, цифра, знак препинания). Выравнивание осуществить, вставляя дополнительные пробелы между словами – равномерно по всей строке.
	
\subsection{Теоритические сведения}
Для того, чтобы перевести из сантиметров в дюймы необходимо количество сантиметров поделить на эквивалент, равный 2,54. Для того чтобы перевести из дюймов в сантиметры - соответственно умножить на 2,54.
	Для выполнения задания использоваилсь функции библиотеки stdio для ввода и вывода.
\subsection{Проектирование}
В ходе проектирования было выделено несколько функций:
\begin{enumerate}
	\item initializText
	Функция выделяет память на массив символов.
	
	\item initializeString 
	Функция выделяет память на строку.
	
	\item inputText
	Функция считывает символы.
	
	\item printText
	Функция печатает текст на экран.
	
	\item getLength
	Функция считает длину строки.
	
	\item countSpaces
	Функция считает пробелы.
	
	\item countChars
	Функция считает символы.
	
	\item getMaxStringLength
	Функция ищет строку с максимальной строкой.
	
	\item insertChar
	
	\item insertChars
	
	\item getString
	
	\item getCharIndex
	
	\item spreadText
	
	\item stringsConsoleUI
\end{enumerate}
\subsection{Описание тестового стенда и методики тестирования}
Среда разработки QtCreator 3.5.0, компилятор Qt 5.5.0 MinGW 32bit, операционная система Ubuntu 14.04.

Для тестирования работы программы были выполнены статический и динамический анализ, также было проведено автоматической тестирование.
\subsection{Тестовый план и результаты тестирования}
Для тестирования программы в подпроект test были добавлены модульные тесты. 
 
 
 
 \begin{tabular}{|c|c|c|}
 	\hline 	
 	Входные данные & Результат & Ожидаемый  результат \\
 	\hline
	3 & 0.39 & 0.39 \\
 	\hline
 	
 \end{tabular}

\subsection{Выводы}
В ходе выполнения я отработала навыки работы с циклом с предусловием. 

%\subsection*{Листинги}
%cm_to_inch
%inch_to_cm
%\lstinputlisting[]
%{../sources/SubDirrrProject/lib/cm_to_inch.c}

%cm_to_inch_console
%\lstinputlisting[]
%{../sources/SubDirrrProject/app/m_to_inch_console_ui.c}


\end{document}